\section{Localization and Explanation}

Automated fault localization \cite{FaultSurvey,Jones2002} provides users with
information about the likely location of a fault in a program's source
code.  \comment{Error explanation \cite{GroceError} also provides information
about the causal structure of a fault --- how it can be avoided, what
triggers it --- a kind of ``fault story.''}  Parnin and Orso recently
suggested that fault localization methods are not actually providing a
great deal of assistance to real users in debugging \cite{AutoHelp}.
Automated fault localization has not been widely adopted in industry.

Parnin and Orso note that most popular methods are
statistical and tend to simply provide an ordered list of statements
to examine.  These statements are not guaranteed to be essentially
related to the fault --- with many techniques, the statements are,
too often, code that executes as a consequence of the fault, code
accidentally executed with the fault, and so forth.  We note that most
evaluations of fault localization have been performed over programs
where it is not clear that debugging is extremely hard, vs. the time
spent applying automated localization and establishing effective
(automated) testing to support automated localization.  Compiler
debugging, on the other hand, especially for subtle optimization-based
problems that produce incorrect code (vs. simple crashes) is generally
known to be difficult.
Researchers reporting subtle compiler bugs often observe the time from
reporting a fault to its correction, even once the fault is assigned,
to be lengthy, and the resulting patches are sometimes thousands of
lines \cite{PLDI13}.  Important compilers, whether JITs where security is paramount
or C compilers used to compile core systems code, also tend to already
be the
subjects of extensive automated testing \cite{jsfunfuzz,csmith}.

\comment{
By definition, when a mutant repairs a test case, it changes the
behavior of the test.  In some cases, this behavioral change is
limited to changes in program state and data values.  However, the
vast majority of mutant repairs also modify the code coverage of a
test; in our experiments, there were no repairs that did not change
coverage.  If a fault can be avoided by, for example, disabling a
problematic optimization in a compiler, this naturally leads to
changed coverage.  If the fault is due to a bad conditional, where a
program path should include additional code, the repair will often
change the execution to include the omitted code.  As a localization
for a fault, then, we can examine the source statements whose
  coverage changed most often in repairs.  Early experiments with this
  approach showed that a useful coverage change localization must  account for the frequency of a
change.  Some coverage changes are extremely common across all
failures, faults, and repairing mutants,
and therefore likely not related to any specific fault.

\begin{quote}
The {\bf Coverage} localization for a failing test case $f \in F$ ($F$
is the set of all failing test cases) ranks
statements by the function $r$ (for rank) where:
%$$U = s : \forall m . \forall t \in F . c(m,t,s) = 1$$
$$C(s) = \sum_{t \in F} (\sum_{m \in R(t)} c(m,t,s))$$
%\[
$$r(s,f) = \sum_{m \in R(f)}\frac{1}{C(s)}c(m,f,s)$$
%\begin{cases}
% \sum_{m \in R(f)}\frac{1}{C(s)}c(m,f,s) & s \not\in U\\
%   0 & s \in U
%\end{cases}
%\]
$c(m,t,s)$ is 1 if coverage of $s$ changes for test $t$
with mutant $m$, and 0 otherwise.  $R(f)$ is the set of mutants repairing $f$.
\end{quote}

That is, the ranking is based on the number of times a statement's
coverage changes its value (covered or not covered) from the original
execution of $f$ in executions of mutants that repair $f$.  Changes are weighted by their inverse frequency over all
failures and all mutants.
Higher values for $r$ result in a higher ranking (more likely to
be faulty).  
%Computing $U$ accounts for the fact that in our
%experiments, a few statements were added or removed for every single
%r
}
Repairing mutants can be used for localization and
explanation.  While few will be equivalent to actual fixes, they
will sometimes overlap the incorrect code, and will often be closely
related to incorrect code, semantically.  Since some tests are
repaired by a large number of mutants, it is important to rank the
mutants to produce a localization.  Based on the onion-ring model proposed above, this is simple: each mutant is ranked as
a localization/explanation based on the \emph{most distant failure it also
  repairs.}  If a given mutant only repairs failures that have similar
repair vectors to the test case being localized, it will be highly
ranked.  If it repairs even one very dissimilar failure, then it will
be considered a poor localization.

\begin{quote}
For a set $F$ of failing test cases and $m$ of mutants, 
the {\bf Repair} localization for $f \in F$, where $R(f) \in m$ is the set
of mutants repairing $f$, ranks
statements by the function $r$:
%$$dmax(m,f) = max_{t : m \in R(t)} d(t,f)$$
\[
r(f) = 
\begin{cases}
\frac{1}{1 + max_{t \in F: m \in R(t)} d(t,f)} & m \in R(f)\\
0 & m \not\in R(f)\\
\end{cases}
\]
%$$\text{leastfar} = m \in R(f) : \neg\exists m' : \text{far}(m',f) <
%\text{far}(m,f)$$
\end{quote}

This ranks a mutant $m$ (and associated statement) by the inverse of the distance
from $f$ to the most distant failing test also repaired by $m$, adding 1
so that if $m$ only repairs failures at distance 0 $r$ is well defined
(and maximal).
If $m$ does not repair $f$, it is not ranked at all.  One advantage of
this approach over some statistical methods is that by
ranking \emph{mutants} the localization offers developers not simply a
ranked statement but a kind of ``explanation'' \cite{GroceError} of the fault:  if \emph{this
mutant} were applied, the fault would not exhibit.  As Kochhar et
al. show \cite{Kochhar}, a large majority (more than 85\%) of developers considered
the ability of a localization tool to provide a \emph{rationale} or
reason for a localization important.  A mutation plus the tests that
stop failing when it is applied is a practical, concrete explanation
of a fault localization.  Mutants that repair a failure provide insight 
into the nature of the fault, and, we hypothesize, more information the 
closer they are to the center of the onion-ring. 

An advantage of this method over some statistical approaches to
localization is that it is not potentially confused in
settings with multiple faults.  Each localization/explanation is based
on a single failing test case, which in most cases fails due to only one
fault.  In fact, other faults (and their failures) theoretically \emph{help} {\bf Repair} rank the mutants.

In general, we expect {\bf Repair} to be used in the context of
compiler (or other complex system) fuzzing, where there are numerous
failing tests, likely from multiple faults.  In such cases, the $f$ to
be localized is presumably selected by FPF.  However, in case our
localization approach is used in a setting where single faults are
expected, we propose an alternative method: let $f$ be an arbitrary
test with as few repairing mutants as possible, to allow the distance
metric to focus attention on a few key mutants. 

\comment{
The coverage localization also provides a kind of explanation, though
a less direct one.  For any
interesting line of code in the coverage-change-based localization,
the user can investigate a particular instance of the change,
by selecting a single mutant and stepping through the execution of the
failing test case in a debugger to the point of change.  The user can also
browse through all mutants that produce this coverage change for the test
case.

Another advantage over many statistical approaches to
localization is that our methods are not potentially confused in
settings with multiple faults.  Each localization/explanation is based
on a single failing test case, which in most cases fails due to only one
fault.  In fact, other faults help both methods rank localizations.}